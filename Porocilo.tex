\documentclass[a4paper, 11pt]{article}
% ukazi za delo s slovenscino -- izberi kodiranje, ki ti ustreza
\usepackage[slovene]{babel}
\usepackage[utf8]{inputenc}
\usepackage[T1]{fontenc}
\usepackage[utf8]{inputenc}
\usepackage{amsmath,amssymb,amsfonts}
\usepackage{url}
\usepackage[normalem]{ulem}
\usepackage[dvipsnames,usenames]{color}
\usepackage{graphicx}
\usepackage{float}
\usepackage{verbatim}
\graphicspath{ {./slike/} }


% ukazi za matematicna okolja
%\theoremstyle{definition} % tekst napisan pokoncno
%\newtheorem{definicija}{Definicija}[section]
%\newtheorem{primer}[definicija]{Primer}
%\theoremstyle{remark}
%\newtheorem*{remark}{Opomba}


%\renewcommand\endprimer{\hfill$\diamondsuit$}


%\theoremstyle{plain} % tekst napisan posevno
%\newtheorem{lema}[definicija]{Lema}
%\newtheorem{izrek}[definicija]{Izrek}
%\newtheorem{trditev}[definicija]{Trditev}
%\newtheorem{posledica}[definicija]{Posledica}


% za stevilske mnozice uporabi naslednje simbole
\newcommand{\R}{\mathbb R}
\newcommand{\N}{\mathbb N}
\newcommand{\Z}{\mathbb Z}
\newcommand{\C}{\mathbb C}
\newcommand{\Q}{\mathbb Q}

% ukaz za slovarsko geslo
\newlength{\odstavek}
\setlength{\odstavek}{\parindent}
\newcommand{\geslo}[2]{\noindent\textbf{#1}\hspace*{3mm}\hangindent=\parindent\hangafter=1 #2}

% naslednje ukaze ustrezno popravi
\newcommand{\program}{Finančna matematika 1.~stopnja} % ime studijskega programa: Matematika/Finan"cna matematika
\newcommand{\imeavtorja}{Patrik Gregorič, Petja Murnik} % ime avtorja
\newcommand{\naslovdela}{Consecutive square packing}
\newcommand{\letnica}{2022} 
\newcommand{\predmet}{Finančni praktikum}

% vstavi svoje definicije ...

%%%%%%%%%%%%%%%%%%%%%%%%%%%%%%%%%%%%%%%%%%%%%%%


\begin{document}

\thispagestyle{empty}
\begin{center}
\begin{minipage}{0.75\linewidth}
    \centering
    {\Large Univerza v Ljubljani \\ \program}
    \\
    \vspace{3cm}

    {\uppercase{\Large \textbf{\naslovdela}}} \\ \predmet\\
    \vspace{3cm}

    {\Large \imeavtorja\par}
    \vspace{9cm}

\end{minipage}
\end{center}

\noindent{\large
Ljubljana, \letnica}
\pagebreak

\thispagestyle{empty}
\tableofcontents
\listoffigures
\pagebreak


\section{Opis problema}

\subsection{Osnovni problem}
Za vsak celo število $i =1, \dots , n$ imamo en kvadrat s stranicami 
dolžine $i$. Želimo najti kvadrat z najkrajšo stranico, v katerega lahko 
zložimo vse kvadrate, ne da bi se notranjosti kvadratov prikrivale (robovi se lahko 
dotikajo). Kvadratov ne moremo obračati lahko jih samo transliramo.

\subsection{Ideja za reševanje}
Problem je enostavno predstavljiv. Imamo zaporedje vse večjih kvadratov, ki jih razporedimo
v ravnino, te se ne prekrivajo, se lahko dotikajo. Kvadratov ne smemo rotirati, lahko
jih le premikamo levo, desno, gor, dol. Omejila sva problem le na pozitivne $x$ in $y$
zaradi lažje predstave. Za majhen $n$ si je problem lahko predstavljati, saj ga
rešimo zelo hitro kar sami.

\begin{figure}[H]
    \centering
    \includegraphics{prvi.png}
    \caption{Lahek primer za $n = 4$}
\end{figure}

Opazimo, da postavimo največja kvadrata drug zraven drugega in ostale le razporedimo s kar
veliko možnostmi na prosta mesta. Problem je enostaven, ker je veliko različnih možnosti.

\section{Reševanje osnovnega problema}
\subsection{Kratek opis}
Za večji $n$ postane problem mnogo težji. Zato sva se odločila reševati s celoštevilskim linearnim programiranjem.
To je: uvedla bova spremenljivke, zapisala omejitve in to poslala skozi vgrajene 
metode za reševanje CLP v programskem jeziku Sage.
\subsection{Zapis problema kot CLP}\label{osnovni CLP}
Za podatke imamo seznam kvadratov. Vsak kvadrat $i$ v seznamu $kvadrati$ je sestavljen
iz trojice $(x-koordinata,y-koordinata,dolzina\_daljice)$, kjer imamo spremenljivko $dolzina\_daljice$
poznano, spremenljivki $x-koordinata,y-koordinata$ pa ne.

 \begin{equation*}
    \begin{array}{ll@{}ll}
    \text{minimize}  & z &\\
    \text{p.p.:}& i[0] \geq 0 & ,  & \text{za $i$ v $kvadrati$}\\
                & i[1] \geq 0 & ,  & \text{za $i$ v $kvadrati$}\\
                & i[0] i[2] \leq z & ,  & \text{za $i$ v $kvadrati$}\\
                & i[1] i[2] \leq z & ,  & \text{za $i$ v $kvadrati$}\\
                & i[0] + i[2] \leq j[0] + M * a[i,j]& , &\text{za $i \neq j $ v $kvadrati$} \\
                & j[0] + j[2] \leq i[0] + M * a[j,i]& , &\text{za $i \neq j $ v $kvadrati$} \\
                & i[1] + i[2] \leq j[1] + M * b[i,j]& , &\text{za $i \neq j $ v $kvadrati$} \\
                & j[1] + j[2] \leq i[1] + M * b[j,i]& , &\text{za $i \neq j $ v $kvadrati$} \\
                & a[i, j] + a[j, i] + b[i, j] + b[j, i] \leq 3& , &\text{za $i \neq j $ v $kvadrati$} \\
                & a[i,j] \in \{0,1\} & , &\text{za $i \neq j $ v $kvadrati$} \\
                & b[i,j] \in \{0,1\} & , &\text{za $i \neq j $ v $kvadrati$} \\
    \end{array}
    \end{equation*}

Pripomnimo nekaj stvari za pojasnilo:
\begin{enumerate}\label{opombice}
    \item  $i[0]$ je $x-koordinata$ ,$i[1]$ je $y-koordinata$ in $i[2]$ je $dolzina\_daljice$ od kvadrata $i$
    \item $a[i,j]$ je spremenljivka, ki pokaže 1, če kvadrat $i$ NI levo od $j$ in 0 sicer
    \item $b[i,j]$ je spremenljivka, ki pokaže 1, če kvadrat $i$ NI spodaj od $j$ in 0 sicer
    \item $M$ je zgornja meja, s katero si pomagamo, da je le nekaj pogojev res. Za njo lahko vzamemo $\displaystyle\sum\limits_{\text{$i$ v }kvadrati}i[2]$
\end{enumerate}



\subsection{Potek reševanja}\label{osnovna koda}
Za podatke imava podane kvadrate s stranicami dolžine $i$, kjer $i = 1, \dots , n$, in $ i \in \Z, n \in \N$.
Ciljna funkcija najinega problema je iskanje $\min z$, kjer je $z$ dolžina stranice največjega kvadrata, v katerega
se zloži vse manjše kvadrate s stranicami $i = 1, \dots , n$.\\
Za reševanje problema sva definirala razred $Kvadrat$.
\begin{verbatim}
from sage.plot.polygon import polygon
class Kvadrat:

    def __init__(self, x, y, dolzina_stranice):
        self.x = x
        self.y = y
        self.dolzina_stranice = dolzina_stranice

    def narisi(self, barva="red"):
        return polygon(
            [(self.x, self.y), (self.x + self.dolzina_stranice, self.y),
             (self.x + self.dolzina_stranice, self.y + self.dolzina_stranice),
             (self.x , self.y + self.dolzina_stranice )],
            color=barva,
            edgecolor='black',
            alpha=0.5,
            zorder=2)

def narisi_rezultat_P(rezultat):
    k = Kvadrat(0,0,rezultat[0])
    rezultat1 = list(rezultat[1])
    K = k.narisi()
    pomoc = []
    for i in rezultat1:
        j = Kvadrat(i[0],i[1],i[2])
        pomoc.append(j)
    RISI = [K]
    for l in pomoc:
        m = l.narisi()
        RISI.append(m)
    show(sum(RISI[i] for i in range(len(RISI))))
\end{verbatim}
Tu sva definirala kvadrat kot levo krajišče $(x,y)$ in dolžino stranice. Drugi dve funkciji
\texttt{narisi} in \verb|narisi_rezultat_P| nam omogočata prikaz rezultatov, ki jih dobiva v obliki seznama.

\begin{verbatim}
def packing(kvadrati): #kvadrati je seznam cifer 
    p = MixedIntegerLinearProgram(maximization=False)
    x = p.new_variable()
    y = p.new_variable()
    zg_meja = sum(kvadrati) + 1
    a = p.new_variable(binary = True) #kvadrat i ni levo od j
    b = p.new_variable(binary = True) #kvadrat j ni desno od j
    p.set_objective(p['z'])
    p.add_constraint((p['z']) >= ((max(kvadrati))+kvadrati[len(kvadrati)-2]))
    for i, d in enumerate(kvadrati): # i je indeks, d je dolžina stranice
        p.add_constraint(x[i] >= 0)  # želimo nenegativne koordinate
        p.add_constraint(y[i] >= 0)
        p.add_constraint(x[i] + d <= p['z']) # omejimo p['z'] z največjo pokrito koordinato
        p.add_constraint(y[i] + d <= p['z'])
        for j , g in enumerate(kvadrati):
            p.add_constraint(x[i] + d <= x[j] + zg_meja * a[i,j])
            p.add_constraint(x[j] + g <= x[i] + zg_meja * a[j,i])
            p.add_constraint(y[i] + d <= y[j] + zg_meja * b[i,j])
            p.add_constraint(y[j] + g <= y[i] + zg_meja * b[j,i])           
    for i, j in Combinations(range(len(kvadrati)), 2):
        p.add_constraint(a[i, j] + a[j, i] + b[i, j] + b[j, i] <= 3)
    z = p.solve()
    xx = p.get_values(x)
    yy = p.get_values(y)
    return (z, [(xx[i], yy[i], d) for i, d in enumerate(kvadrati)])
\end{verbatim}

\noindent Ta CLP nam reši problem tudi za večji $n$, do katere rešitve ne pridemo tako intuitivno. Funkcija 
sprejme seznam cifer v obliki $range(1, n +1)$ in vrne $tuple$, ki vsebuje za prvi element dolžino
največjega kvadrata, torej rešitev problema, naslednji element je pa seznam kvadratov, ki so podani od
največjega do najmanjšega. Program najprej preveri, da so vse koordinate nenegativne, saj sva tako definirala
problem. Nato preverimo pogoje, da morajo biti vsi kvadrati znotraj največjega, ki ga želimo minimizirat in še, da 
se noben od njih ne sme prekrivat. S pomočjo teh omejitev nam program poda pravilno rešitev, ki jo nato še prikaževa s
funkcijo \verb|narisi_rezultat_P|.

\begin{figure}[H]
    \centering
    \includegraphics{Lahek_primer.png}
    \caption{Primer za $n = 7$}
\end{figure}

\noindent Opazimo, da problem hitro postane bolj kompleksen za reševanje.

\section{Nadgradnja problema}
\subsection{Opis nadgradnje}
V nadeljevanju bova problem tudi reševala v primeru, 
če namesto enega kvadrata z dolžino 
stranice vzamemo dva/tri/štiri \dots take kvadrate. 
Obravnavala bova tudi primer, ko je lahko ena točka v ravnini (velikem kvadratu) pokrita z 
dvema/tremi/štirimi \dots  kvadrati.
\subsection{Zapis problema kot CLP, ko imamo kvadrate ne nujno zaporednih velikost}
CLP v tem problemu izgleda identičen tistemu v osnovnem primeru \ref{osnovni CLP}

\subsection{Potek reševanja, ko imamo kvadrate ne nujno zaporednih velikost}
Dani podatki, ki jih sedaj ponudimo je seznam dolžin daljic kvadratov. V tem primeru ni 
pomembno, koliko je katerih dolžin, važno je le, da je seznam urejen. To je tako,
da so dolžine stranic naraščujoče.
V tem primeru, nam funkcija \verb|packing()| kot pri osnovnem primeru v \ref{osnovna koda}, na danem seznamu deluje pravilno in nam najde rešitev.
Nisva se ubadala s pisanjem funkcije, ki bi generirala seznam, ki bi imel števila od 
$1$ do $n$ in vsako $l$-krat. To lahko bralec naredi sam brez večjih težav in preveri, da 
stvar deluje.

\begin{figure}[H]
    \centering
    \includegraphics{Vec_istih_kvadratov.png}
    \caption{Primer za pet kvadratov dolžine $1$ in dva dolžine $2$}
\end{figure}

\subsection{Zapis problema kot CLP, ko je vsaka točka v prostoru lahko pokrita $k$-krat}
Za podatke imamo seznam kvadratov in število $k$. Vsak kvadrat $i$ v seznamu $kvadrati$ je sestavljen
iz trojice $(x-koordinata,y-koordinata,dolzina\_daljice)$, kjer imamo spremenljivko $dolzina\_daljice$
poznano, spremenljivki $x-koordinata,y-koordinata$ pa ne. Število $k$ nam pove, kolikokrat je
lahko ena točka v prostoru pokrita, tj. v koliko kvadratih je lahko.

\begin{equation*}
    \begin{array}{ll@{}ll}
    \text{minimize}  & z &\\
    \text{p.p.:}& i[0] \geq 0 & ,  & \text{za $i$ v $kvadrati$}\\
                & i[1] \geq 0 & ,  & \text{za $i$ v $kvadrati$}\\
                & i[0] +i[2] \leq z & ,  & \text{za $i$ v $kvadrati$}\\
                & i[1]  + i[2] \leq z & ,  & \text{za $i$ v $kvadrati$}\\
                & i[0] + i[2] \leq j[0] + M * a[i,j]& , &\text{za $i \neq j $ v $kvadrati$} \\
                & j[0] + j[2] \leq i[0] + M * a[j,i]& , &\text{za $i \neq j $ v $kvadrati$} \\
                & i[1] + i[2] \leq j[1] + M * b[i,j]& , &\text{za $i \neq j $ v $kvadrati$} \\
                & j[1] + j[2] \leq i[1] + M * b[j,i]& , &\text{za $i \neq j $ v $kvadrati$} \\
                & a[i, j] + a[j, i] + b[i, j] + b[j, i] \leq 3 + L[i,j]& , &\text{za $i \neq j $ v $kvadrati$} \\
                & \displaystyle\sum_{\substack{g \in {1,\dots , k+1}\\ f \ge g, g\in {1, \dots k+1}}} L[i_g,i_f] \leq {k+1 \choose 2} -1& , &\text{za $i_1 \neq i_2\neq \dots \neq i_{k+1} $ v $kvadrati$} \\ 
                & a[i,j] \in \{0,1\} & , &\text{za $i \neq j $ v $kvadrati$} \\
                & b[i,j] \in \{0,1\} & , &\text{za $i \neq j $ v $kvadrati$} \\
                & L[i,j] \in \{0,1\} & , &\text{za $i \neq j $ v $kvadrati$}
    \end{array}
    \end{equation*}

Večino pripomb, ki so tu potrebne, smo že pojasnili pri osnovnem primeru \ref{opombice}. Dodati je treba samo še eno:
\begin{enumerate}
    \item Če $L[i,j] = 1$, pomeni da se kvadrata i in j sekata. Pogoj z vsoto pa nam zagotovi da se nam ne seka preveč kvadratov. 
    Če izberemo $k+1$ kvadrat in je vsota enaka ${k+1 \choose 2}$, tedaj se vsaka dva kvadrata iz izbranih $k+1$ kvadratov sekata.
    Torej obstaja točka ki je v vseh teh kvadratih in torej ne zadošča pogoju.
\end{enumerate}    

\subsection{Potek reševanja, ko je vsaka točka v prostoru lahko pokrita $k$-krat}
Podatki, ki jih v tem primeru dobimo je število $k$,  ki pove kolikokrat je 
lahko ena točka v prostoru pokrita in urejen seznam dolžin stranic kvadratov.
Za ta primer sva definirala naslednjo funkcijo, ki izračuna željen razultat:

\begin{verbatim}
    def packing_ktimes(kvadrati,k): #kvadrati je seznam cifer, tale dela če so cifre urejene in vsako točko lahko pokrijemo k-krat
    p = MixedIntegerLinearProgram(maximization=False,solver='GLPK')
    x = p.new_variable()
    y = p.new_variable()
    zg_meja = sum(kvadrati) + 1
    a = p.new_variable(binary = True) #kvadrat i ni levo od j
    b = p.new_variable(binary = True) #kvadrat j ni desno od j
    L = p.new_variable(binary = True)
    p.set_objective(p['z'])
    p.add_constraint((p['z']) >= (max(kvadrati)))
    for i, d in enumerate(kvadrati): # i je indeks, d je dolžina stranice
        p.add_constraint(x[i] >= 0)  # želimo nenegativne koordinate
        p.add_constraint(y[i] >= 0)
        p.add_constraint(x[i] + d <= p['z']) # omejimo p['z'] z največjo pokrito koordinato
        p.add_constraint(y[i] + d <= p['z'])
        for j , g in enumerate(kvadrati):
            p.add_constraint(x[i] + d <= x[j] + zg_meja * a[i,j])
            p.add_constraint(x[j] + g <= x[i] + zg_meja * a[j,i])
            p.add_constraint(y[i] + d <= y[j] + zg_meja * b[i,j])
            p.add_constraint(y[j] + g <= y[i] + zg_meja * b[j,i])
    for i, j in Combinations(range(len(kvadrati)), 2):
        p.add_constraint(a[i, j] + a[j, i] + b[i, j] + b[j, i] <= 3 + L[i,j])
    for C in Combinations(range(len(kvadrati)), (k+1)):
        p.add_constraint(sum(sum(L[C[i],C[j]] for j in range(i+1,k+1)) for i in range(k +1 )) <= comb((k+1),2)-1)
    z = p.solve()
    xx = p.get_values(x)
    yy = p.get_values(y)
    return (z, [(xx[i], yy[i], d) for i, d in enumerate(kvadrati)]) 
\end{verbatim}

\begin{figure}[H]
    \centering
    \includegraphics{Dvakratno_prekrivanje.png}
    \caption{Primer, ko je vsaka točka lahko pokrita $2-krat$}
\end{figure}

\noindent Tak problem program reši bistveno hitreje, saj ima več možnosti za ureditev kvadratov. Večkrat,
ko je vsaka točka lahko pokrita enostavnejši postane izračun rezultata, saj si lahko predstavljamo problem,
da vsak kvadrat razdelimo na svoj podproblem in ga rešujemo kot osnovni problem z malo več svobode pri
omejevanju mej.

\section{Sklep}
Problem je težek, saj ne obstaja enolična rešitev. Program ima na voljo veliko možnosti za rešitve, hkrati
pa mora biti problem dobro zastavljen, da program dobi pravilno rešitev. Za izračun si moramo postaviti tesne
meje, ki omejujejo dopustne rešitve za hitrost delovanja programa.\\
V nadgradnji problema opazimo veliko hitrejše delovanje programa. Zahteve niso tako omejujoče s prekrivanjem 
ali pa z velikostjo kvadratov. Pri večkratnem prekrivanju se problem za program zelo poenostavi, kar močno
vpliva na hitrost.\\
Pri zlaganju kvadratov v kvadrat opazimo veliko nepokritih točk, kar nam poda idejo, da bi bilo bolj
učinkovito zlagati kvadrate v pravokotnik.

\end{document}