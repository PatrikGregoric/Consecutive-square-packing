\documentclass[12pt,a4paper]{amsart}
% ukazi za delo s slovenscino -- izberi kodiranje, ki ti ustreza
\usepackage[slovene]{babel}
\usepackage[utf8]{inputenc}
%\usepackage[T1]{fontenc}
%\usepackage[utf8]{inputenc}
\usepackage{amsmath,amssymb,amsfonts}
\usepackage{url}
%\usepackage[normalem]{ulem}
\usepackage[dvipsnames,usenames]{color}

% ne spreminjaj podatkov, ki vplivajo na obliko strani
\textwidth 15cm
\textheight 24cm
\oddsidemargin.5cm
\evensidemargin.5cm
\topmargin-5mm
\addtolength{\footskip}{10pt}
\pagestyle{plain}
%\overfullrule=15pt % oznaci predlogo vrstico


% ukazi za matematicna okolja
\theoremstyle{definition} % tekst napisan pokoncno
\newtheorem{definicija}{Definicija}[section]
\newtheorem{primer}[definicija]{Primer}
\newtheorem{opomba}[definicija]{Opomba}

\renewcommand\endprimer{\hfill$\diamondsuit$}


\theoremstyle{plain} % tekst napisan posevno
\newtheorem{lema}[definicija]{Lema}
\newtheorem{izrek}[definicija]{Izrek}
\newtheorem{trditev}[definicija]{Trditev}
\newtheorem{posledica}[definicija]{Posledica}


% za stevilske mnozice uporabi naslednje simbole
\newcommand{\R}{\mathbb R}
\newcommand{\N}{\mathbb N}
\newcommand{\Z}{\mathbb Z}
\newcommand{\C}{\mathbb C}
\newcommand{\Q}{\mathbb Q}

% ukaz za slovarsko geslo
\newlength{\odstavek}
\setlength{\odstavek}{\parindent}
\newcommand{\geslo}[2]{\noindent\textbf{#1}\hspace*{3mm}\hangindent=\parindent\hangafter=1 #2}

% naslednje ukaze ustrezno popravi
\newcommand{\program}{Finančna matematika 1.~stopnja} % ime studijskega programa: Matematika/Finan"cna matematika
\newcommand{\imeavtorja}{Patrik Gregorič, Petja Murnik} % ime avtorja
\newcommand{\naslovdela}{Consecutive square packing}
\newcommand{\letnica}{2022} 
\newcommand{\predmet}{Finančni praktikum}

% vstavi svoje definicije ...

%%%%%%%%%%%%%%%%%%%%%%%%%%%%%%%%%%%%%%%%%%%%%%%


\begin{document}

\thispagestyle{empty}
\noindent{\large
UNIVERZA V LJUBLJANI\\[1mm]
FAKULTETA ZA MATEMATIKO IN FIZIKO\\[5mm]
\program
\vfill

\begin{center}{\large
\imeavtorja\\[2mm]
{\bf \naslovdela}\\[10mm]
\predmet\\[2mm] 
}
\end{center}
\vfill

\noindent{\large
Ljubljana, \letnica}
\pagebreak

\thispagestyle{empty}
\tableofcontents
\pagebreak


\section{Opis problema}
\subsection{Osnovni problem}
Za vsak celo število $i =1, \dots , n$ imamo en kvadrat s stranicami 
dolžine $i$. Želimo najti kvadrat z najkrajšo stranico, v katerega lahko 
zložimo vse kvadrate, ne da bi se notranjosti kvadratov prikrivale(robovi se lahko 
dotikajo). Kvadratov ne moremo obračati lahko jih samo transliramo. 

\subsection{Nadgradnja problema}
V nadeljevanju bova problem tudi reševala v primeru, 
če namesto enega kvadrata z dolžino 
stranice vzamemo dva/tri/štiri \dots take kvadrate. 
Obravnavala bova tudi primer, ko je lahko ena točka v ravnini(velikem kvadratu) pokrita z 
dvema/tremi/štirimi \dots  kvadrati. 




\end{document}